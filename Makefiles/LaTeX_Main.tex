\documentclass[UTF8,titlepage]{ctexart}
\usepackage{hyperref}
\usepackage[a4paper]{geometry}
\usepackage{minted}
\usepackage{python}
\usemintedstyle{bw}
\usepackage{amsmath}
\usepackage{amssymb}
\usepackage{amsthm}
\usepackage{bm}
\usepackage{siunitx}
\usepackage{color}
\usepackage{fancyhdr}
\usepackage{float}
\usepackage{subcaption}
\usepackage{enumitem}
\usepackage[]{graphicx}
\usepackage{booktabs}
%\usepackage{pifont} % this is to provide macro \ding in line 12
%
%% \num{10e2}
%% \mmHg
%% \SI{4}{\milli\meter}
%% \gg  % instead of $>\!\!>$
%
%
%%% For footnote style
%\usepackage[perpage]{footmisc}
%\renewcommand\thefootnote{\ding{\numexpr171+\value{footnote}}}

\pagestyle{plain}
\usemintedstyle{borland}
%\geometry{a4paper, bottom=4cm}
%\graphicspath{{}}
\bibliographystyle{plain}

\hypersetup{%
        colorlinks=false,
        pdftitle={},
        pdfauthor={Z. Y. Liu}}

\newcommand{\me}{\mathrm{e}}
\newcommand{\mi}{\mathrm{i}}
\newcommand{\mRe}{\mathit{Re}}
\newcommand{\mSt}{\mathit{St}}
\newcommand{\mRo}{\mathit{Ro}}
\newcommand{\Karman}{K\'arm\'an}
%\newcommand{\diff}{\,\mathrm{d}}
\DeclareMathOperator\dif{d\!}

\title{}
\author{}
\date{\today}

\begin{document}
\maketitle


%% BiBTeX settings
%\bibliography{tex}
\end{document}

\if\false%
%% How to add figures side by side?
\begin{figure}
\centering
\begin{subfigure}{.5\textwidth}
  \centering
  \includegraphics[width=.4\linewidth]{image1}
  \caption{A subfigure}
\label{fig:sub1}
\end{subfigure}%
\begin{subfigure}{.5\textwidth}
  \centering
  \includegraphics[width=.4\linewidth]{image1}
  \caption{A subfigure}
\label{fig:sub2}
\end{subfigure}
\caption{A figure with two subfigures}
\label{fig:test}
\end{figure}

\begin{figure}
\centering
\begin{minipage}{.5\textwidth}
  \centering
  \includegraphics[width=.4\linewidth]{image1}
  \captionof{figure}{A figure}
\label{fig:test1}
\end{minipage}%
\begin{minipage}{.5\textwidth}
  \centering
  \includegraphics[width=.4\linewidth]{image1}
  \captionof{figure}{Another figure}
\label{fig:test2}
\end{minipage}
\end{figure}

%% The simple manual for minted package
\begin{minted}
  [
    frame=lines,
    framesep=2mm,
    baselinestretch=1.2,
    bgcolor=LightGray,
    fontsize=\footnotesize,
    linenos
  ]
  {python}
  % some Python code
\end{minted}

\inputminted[firstline=2, lastline=12]{octave}{BitXorMatrix.m}

\mint{html}|<h2>Something <b>here</b></h2>|

%% End %% Actually there are many fabulous features...
\fi
